% Created 2023-03-12 Sun 17:16
% Intended LaTeX compiler: pdflatex
\documentclass[11pt]{article}
\usepackage[utf8]{inputenc}
\usepackage[T1]{fontenc}
\usepackage{graphicx}
\usepackage{longtable}
\usepackage{wrapfig}
\usepackage{rotating}
\usepackage[normalem]{ulem}
\usepackage{amsmath}
\usepackage{amssymb}
\usepackage{capt-of}
\usepackage{hyperref}
\author{Yusheng Zhao}
\date{\today}
\title{Final Essay Proposal}
\hypersetup{
 pdfauthor={Yusheng Zhao},
 pdftitle={Final Essay Proposal},
 pdfkeywords={},
 pdfsubject={},
 pdfcreator={Emacs 28.2 (Org mode 9.6)}, 
 pdflang={English}}
\begin{document}

\maketitle
\tableofcontents

The following content is generate by ChatGPT. The main gist of my final
presentation is to extend/give an example of our learning on topological
invariants. This is achieved by discussion of emergence of Majorana zero modes
in superconducting systems. Specifically, the 1D toy model that is presented in
this \href{https://arxiv.org/pdf/1202.1293.pdf}{paper.}

The emergence of Majorana fermions in superconducting systems is related to the
concept of a topological invariant or topological number. A topological
invariant is a quantity that characterizes the topological properties of a
system and remains unchanged even under small perturbations. In the case of
Majorana fermions, their existence is tied to a topological invariant known as
the Chern number or the topological index.

In 1D superconducting systems, the emergence of Majorana fermions is related to
the topological invariant known as the winding number, which characterizes the
topological properties of the system. The winding number describes how many
times the superconducting phase changes around the wire, and it is related to
the number of Majorana fermions that can emerge at the ends of the wire.
Specifically, the number of Majorana fermions is equal to the absolute value of
the winding number.

In 2D superconducting systems, the emergence of Majorana fermions is related to
the topological invariant known as the Chern number, which characterizes the
topological properties of the system in terms of the curvature of the energy
bands. The Chern number is related to the number of Majorana fermions that can
form at the edge of the superconductor, and specifically, the number of Majorana
modes is equal to the absolute value of the Chern number.

The existence of Majorana fermions in superconducting systems is therefore
closely tied to the topological properties of the system, as described by
topological invariants such as the winding number and the Chern number. These
topological invariants provide a robust and quantitative way to characterize the
presence and properties of Majorana fermions in these systems.
\end{document}