% Created 2023-05-10 Wed 10:21
% Intended LaTeX compiler: pdflatex
\documentclass[presentation]{beamer}
\usepackage[utf8]{inputenc}
\usepackage[T1]{fontenc}
\usepackage{graphicx}
\usepackage{longtable}
\usepackage{wrapfig}
\usepackage{rotating}
\usepackage[normalem]{ulem}
\usepackage{amsmath}
\usepackage{amssymb}
\usepackage{capt-of}
\usepackage{hyperref}
\usepackage{braket}
\usepackage{listings}
\usepackage{bbm}
\setbeameroption{show notes}
\usetheme{Berlin}
\usefonttheme{professionalfonts}
\author{Yusheng Zhao}
\date{2023-05-10}
\title{Physics and Computation}
\subtitle{Topologically protected qubits}
\hypersetup{
 pdfauthor={Yusheng Zhao},
 pdftitle={Physics and Computation},
 pdfkeywords={},
 pdfsubject={AMAT 5600 Final Presentation},
 pdfcreator={Emacs 28.2 (Org mode 9.6.1)}, 
 pdflang={English}}
\begin{document}

\maketitle
\begin{frame}{Outline}
\tableofcontents
\end{frame}


\section{Motivation}
\label{sec:orge439c6f}
\begin{frame}[label={sec:org48a6d48}]{My interest: Quantum Computer}
\begin{itemize}
\item \alert{Motivation}: quantum advantage on some problems
\item How does it differ from classical one?
\begin{enumerate}
\item Information encoded on a qubit: \(\ket{\psi} = \alpha \ket{0} + e^{i\theta}
     \beta \ket{1}\)
\item \(\ket{0}\) and \(\ket{1}\) are usually degrees of freedom of underlying
\alert{physical} system
\end{enumerate}
\end{itemize}
\end{frame}

\note{Note
\begin{itemize}
\item Let me first introduce quantum computer for those who are not familiar
\item Qubit vs Bit
\item Quantum Computers could solve some problems more efficiently compared to
classical computers. For example, Shor's algorithm is able to factor large
integers \(N\) while providing an exponential speed up compare to classical one.
\end{itemize}}

\begin{frame}[label={sec:orgad1512f}]{Noise in Quantum Computer}
\begin{itemize}
\item Noise is the archenemy
\item \alert{Example}: Imp and abucus
\item Caused by unwanted physical interactions
\end{itemize}

\footnotetext{We refer to all noise loosely by decoherence.}
\end{frame}
\note{Note
\begin{itemize}
\item However, this technology is plagued by noise. Roughly speaking, noise is like
a little daemon who flips the abucus that you use to do calculation. The
source of those noise come from unwanted physical interaction. or even badly
calibrated actions.
\item Loss of phase: \(a^{\dagger}_{j}a_{j} = (1+ i \gamma_{2j-1}\gamma)/2\)
\end{itemize}}

\begin{frame}[label={sec:orgd2522a4}]{Observation and Solution}
\begin{itemize}
\item \alert{Observation}: Physical interactions are local
\item \alert{Proposition}: Design a qubit to store information \alert{non-locally}
\item ``If a physical system were to have quantum topological (necessarily nonlocal)
degrees of freedom, which were insensitive to local probes, then information
contained in them would be automatically protected against errors caused by
local interactions with the environment.'' - A. Kitaev
\end{itemize}
\end{frame}

\note{Note
\begin{itemize}
\item This gives us an idea. Since all known physical interactions are local, could
be store our information non-locally to alleviate the effect of noise?
\item The following quote of Alexi Kitaev is too good to be missed
\end{itemize}}

\section{Topological Protection}
\label{sec:org63cd030}

\begin{frame}[label={sec:org2e054d2}]{Topological Invariant}
\begin{center}
\includegraphics[width=.9\linewidth]{./TopoInvariant.png}
\end{center} [Prof. Li Slides]
\end{frame}

\note{Note
\begin{itemize}
\item This is very abstract, can we get a better intuition from what we have learned
in class?
\item Consider Topological Invariant
\item Recall our definition of a topological invariant, as previously defined in
class. Both the Gauss-Bonnet theorem and the calculation of the Berry phase
require an integral over the entire system. Local information alone is
insufficient to determine the topological invariants of a system. Similarly,
if one has the ability to alter a system locally, they cannot alter the
topological invariant number of that system.
\end{itemize}}

\begin{frame}[label={sec:org69dd94c}]{Topological Degeneracy}
\begin{itemize}
\item Defined on low temperature gapped quantum many body system
\item Degenerate ground state emerges after phase transition
\item Arises from topologically non-trivial phase
\item Protected from local perturbation
\item Encode qubit onto these d.o.f
\end{itemize}
\end{frame}

\note{Note
\begin{itemize}
\item Using this as a guide, we turn our attention to topological systems. In the
limit of large system size, a topologically non-trivial phase can emerge from
a gapped quantum many-body system. One key feature of the non-trivial phase is
that it possesses topologically degenerate ground states that are not
degenerate in the topologically trivial phase. Therefore, the degeneracy of
the ground states is a direct consequence of the topology of the system's
phase. Applying the logic from the previous paragraph, we expect these ground
states to be protected from local noise, meaning decoherence should not affect
them.
\end{itemize}}

\section{Kitaev's Toy Model}
\label{sec:org77c79a6}
\begin{frame}[label={sec:org1238688}]{Hamiltonian \cite{kitaevUnpairedMajoranaFermions2001}}
\begin{itemize}
\item \(H = \sum_{n=1}^{N} [- \mu (a^{\dagger}_{n}a_{n}- \frac{1}{2}) - w
  (a^{\dagger}_{n}a_{n+1} + a^{\dagger}_{n+1}a_{n}) + \Delta a_{n}a_{n+1} +
  \Delta^{*}a^{\dagger}_{n+1}a^{\dagger}_{n})]\)
\item 1D chain of fermions
\item \(\mu\): chemical potential
\item \(w\): hopping strength
\item \(\Delta\): induced superconducting gap
\end{itemize}
\end{frame}

\begin{frame}[label={sec:org0d1330a}]{Emergence of Non-trivial Phase \cite{huangIntroductionMajoranaZero2021}}
\begin{itemize}
\item \(|\Delta| = w > 0\)
\item \(a_{n} = \frac{1}{2}(e^{-i \theta /2}\gamma_{2n} +
  e^{i\theta/2}\gamma_{2n-1})\)
\item \(\gamma\) is the Majorana creation/anhilation operator
\item \(\tilde{a}_{n} = (\gamma_{2n}-i\gamma_{2n+1})/2\)
\item \alert{Diagonalize}: \(H = 2w \sum_{n=1}^{N-1}(\tilde{a}^{\dagger}_{n}\tilde{a}_{n} -1/2)\)
\end{itemize}
\end{frame}

\begin{frame}[label={sec:orgb9a864b}]{A picture is worth a thousand words}
\begin{center}
\includegraphics[width=.9\linewidth]{./two-phases.png}
\end{center}
\end{frame}

\note{Note
\begin{itemize}
\item Note, \(\gamma_1\) and \(\gamma_{2N}\) are not in Hamiltonian
\item Have zero energy.
\item Combine to make fermonic mode \(\tilde{a}_{0} =(\gamma_{1}+i\gamma_{2N})/2\)
\item \(\ket{0}\) and \(\ket{1}\) of above creation operator have degenerate energy.
\item Also protected by topology. Can be made into protected qubits!
\end{itemize}}

\section{Take Home Message}
\label{sec:org28e5c75}
\begin{frame}[label={sec:orgb56f5bc}]{Physics and Computation}
\begin{itemize}
\item ``Information is Physical'' \cite{landauerThereAreNo1991}
\item Topologically degenerate degree of freedom sees not local perturbation
\end{itemize}
\footnotetext{*MZM* is not Majorana fermion.}
\end{frame}

\note{Note
\begin{itemize}
\item Information is physical, meaning that the effecacy of the computation relies
very much so on the system that realizes it. Computation is not merely
something on the paper. It's very much so related to the physical world.
\item Topological degree of freedom is calculated from the system-wide point of
view. Therefore, it could not be probed locally hence it's immune to local
error.
\end{itemize}}


\section{Bibliography}
\label{sec:orgb4256f7}
\begin{frame}[allowframebreaks]{References}
\bibliographystyle{apalike}
\bibliography{finalp}
\end{frame}
\end{document}