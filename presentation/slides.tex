% Created 2023-05-01 Mon 22:07
% Intended LaTeX compiler: pdflatex
\documentclass[presentation]{beamer}
\usepackage[utf8]{inputenc}
\usepackage[T1]{fontenc}
\usepackage{graphicx}
\usepackage{longtable}
\usepackage{wrapfig}
\usepackage{rotating}
\usepackage[normalem]{ulem}
\usepackage{amsmath}
\usepackage{amssymb}
\usepackage{capt-of}
\usepackage{hyperref}
\usepackage{braket}
\usepackage{listings}
\usepackage{bbm}
\setbeameroption{show notes}
\usetheme{Berlin}
\usefonttheme{professionalfonts}
\author{Yusheng Zhao}
\date{2023-05-10}
\title{Topological Order and Noise}
\subtitle{Emergence of Majorana Zero Mode}
\hypersetup{
 pdfauthor={Yusheng Zhao},
 pdftitle={Topological Order and Noise},
 pdfkeywords={},
 pdfsubject={AMAT 5600 Final Presentation},
 pdfcreator={Emacs 28.2 (Org mode 9.6.1)}, 
 pdflang={English}}
\begin{document}

\maketitle
\begin{frame}{Outline}
\tableofcontents
\end{frame}


\section{Motivation}
\label{sec:orgc9fb8d5}
\begin{frame}[label={sec:org0d383f2}]{Noise in Quantum Computer}
\begin{itemize}
\item Quantum Computing has advantage over Classical
\item Noise is the archenemy
\item Solution: store information \alert{non-locally}
\item ``If a physical system were to have quantum topological (necessarily nonlocal)
degrees of freedom, which were insensitive to local probes, then information
contained in them would be automatically protected against errors caused by
local interactions with the environment.'' - A. Kitaev
\end{itemize}
\end{frame}
\note{Note
\begin{itemize}
\item Quantum Computers could solve some problems more efficiently compared to
classical computers. For example, Shor's algorithm is able to factor large
integers \(N\), in \(\mathcal{O}(log(N)^{2}loglog(N)\) time. Meanwhile, the best
known classical algorithm is \(\mathcal{O}(e^{1.9
  log(N)^{1/3}loglog(N)^{2/3}}\).
\item However, this technology is plagued by noise. Roughly speaking, noise is like
a little daemon who flips the abucus that you use to do calculation. The
source of those noise come from unwanted physical interaction. or even badly
calibrated actions.
\item For the purpose of this talk, we focus on unwanted physical interaciton.
\item This gives us an idea. Since all known physical interactions are local, could
be store our information non-locally to alleviate the effect of noise?
\end{itemize}}

\section{Emergence of Majorana Zero Mode}
\label{sec:org18c5f16}

\begin{frame}[label={sec:org5d03911}]{Kitaev's Toy Model}
\end{frame}

\begin{frame}[label={sec:orgea69135}]{Topological Order}
\begin{itemize}
\item Where the hell did it emerge?
\end{itemize}
\end{frame}

\section{Take Home Message}
\label{sec:org31d406f}
\begin{frame}[label={sec:org722364f}]{Physics and Computation}
\begin{itemize}
\item ``Information is Physical'' - R. Landauer
\item Topological degree of freedom sees not local perturbation
\end{itemize}
\end{frame}

\note{Note
\begin{itemize}
\item Information is physical, meaning that the effecacy of the computation relies
very much so on the system that realizes it. Computation is not merely
something on the paper. It's very much so related to the physical world.
\item Topological degree of freedom is calculated from the system-wide point of
view. Therefore, it could not be probed locally hence it's immune to local
error.
\end{itemize}}


\section{Bibliography}
\label{sec:orgfdbd893}
\begin{frame}[allowframebreaks]{References}
\bibliographystyle{apalike}
\bibliography{finalp}
\end{frame}
\end{document}