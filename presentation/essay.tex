% Created 2023-05-05 Fri 16:14
% Intended LaTeX compiler: pdflatex
\documentclass[11pt]{article}
\usepackage[utf8]{inputenc}
\usepackage[T1]{fontenc}
\usepackage{graphicx}
\usepackage{longtable}
\usepackage{wrapfig}
\usepackage{rotating}
\usepackage[normalem]{ulem}
\usepackage{amsmath}
\usepackage{amssymb}
\usepackage{capt-of}
\usepackage{hyperref}
\author{Yusheng Zhao}
\date{\today}
\title{Physics and Computation}
\hypersetup{
 pdfauthor={Yusheng Zhao},
 pdftitle={Physics and Computation},
 pdfkeywords={},
 pdfsubject={},
 pdfcreator={Emacs 28.2 (Org mode 9.6.1)}, 
 pdflang={English}}
\begin{document}

\maketitle

\section{Disclaimer}
\label{sec:orge7447df}
Yes, chatgpt essay writing skill is a piece of shit. But I used it to proof-read
and polish my essay.

\section{Quantum Computation: A plagued technology}
\label{sec:org89157d0}
Quantum computers possess the capability to solve certain problems with greater
efficiency than classical computers. Shor's algorithm, for instance, can factor
large integers \(N\) in \(\mathcal{O}(log(N)^2 loglog(N))\) time, whereas the most
efficient classical algorithm is \(\mathcal{O}(e^(1.9 log(N)^(1/3)
loglog(N)^(2/3)))\) \cite{ShorAlgorithm2023}. Consequently, quantum computers have
the potential to provide exponential speedup for factoring large integers.
However, modern encryption techniques that rely on the difficulty of factoring
integers are unlikely to be compromised anytime soon.

One of the most significant challenges in quantum computing is noise. In
essence, noise refers to a disruptive force that perturbs the qubits used for
computation. It is analogous to a mischievous imp that upsets the abacus one
uses for arithmetic operations. In other words, computation results from a noisy
quantum computer are no longer precise. These noises arise from unwanted
physical interactions between the qubits undergoing coherent evolution and their
environment or poorly calibrated operations used for computation.

Although techniques such as quantum error correction code can protect data in a
fault-tolerant quantum memory, they rely on relatively good physical
operations\cite{girvinIntroductionQuantumError2022}. Presently, our technology is
still struggling to keep up. Consequently, there is a need to develop a
naturally robust computing approach that is immune to unwanted interactions
causing noise.

\section{Topological Invariant: Physical Motivation}
\label{sec:org666bd8e}
As Rolf Landauer said, ``Information is Physical''. When we do computation, we
first encode the information on physical systems and evolve those systems to
carry out computation.

\section{Majorana Zero Mode: Physical Realization}
\label{sec:orgcc98196}

\section{Bibliography}
\label{sec:org079d476}
\subsection{References}
\label{sec:orgaa29ef4}
\bibliographystyle{apalike}
\bibliography{finalp}
\end{document}