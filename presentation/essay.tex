% Created 2023-05-07 Sun 14:43
% Intended LaTeX compiler: pdflatex
\documentclass[11pt]{article}
\usepackage[utf8]{inputenc}
\usepackage[T1]{fontenc}
\usepackage{graphicx}
\usepackage{longtable}
\usepackage{wrapfig}
\usepackage{rotating}
\usepackage[normalem]{ulem}
\usepackage{amsmath}
\usepackage{amssymb}
\usepackage{capt-of}
\usepackage{hyperref}
\usepackage{braket}
\author{Yusheng Zhao}
\date{\today}
\title{Physics and Computation}
\hypersetup{
 pdfauthor={Yusheng Zhao},
 pdftitle={Physics and Computation},
 pdfkeywords={},
 pdfsubject={},
 pdfcreator={Emacs 28.2 (Org mode 9.6.1)}, 
 pdflang={English}}
\begin{document}

\maketitle

\section{Disclaimer}
\label{sec:orgaecc385}
Yes, chatgpt is good at spitting out nonsense. But I used it to proof-read and
polish my essay.

\section{Quantum Computation: A plagued technology}
\label{sec:orga2439d9}
Quantum computers possess the capability to solve certain problems with greater
efficiency than classical computers. Shor's algorithm, for instance, can factor
large integers \(N\) in \(\mathcal{O}(log(N)^2 loglog(N))\) time, whereas the most
efficient classical algorithm is \(\mathcal{O}(e^(1.9 log(N)^(1/3)
loglog(N)^(2/3)))\)\cite{ShorAlgorithm2023}. Consequently, quantum computers have the potential to
provide exponential speedup for factoring large integers. However, modern
encryption techniques that rely on the difficulty of factoring integers are
unlikely to be compromised anytime soon.

One of the most significant challenges in quantum computing is noise. In
essence, noise refers to a disruptive force that perturbs the qubits used for
computation. It is analogous to a mischievous imp that upsets the abacus one
uses for arithmetic operations. In other words, computation results from a noisy
quantum computer are no longer precise. These noises arise from unwanted
physical interactions between the qubits undergoing coherent evolution and their
environment or poorly calibrated operations used for computation.

Although techniques such as quantum error correction code can protect data in a
fault-tolerant quantum memory, they rely on relatively good physical
operation\cite{girvinIntroductionQuantumError2022}. Presently, our technology is
still struggling to keep up\cite{kitaevUnpairedMajoranaFermions2001}.
Consequently, there is a need to develop a naturally robust computing approach
that is immune to unwanted interactions causing noise.

\section{Topology: A Global Picture}
\label{sec:orgadefa82}
As Rolf Landauer famously stated, ``Information is
Physical.'' \cite{landauerThereAreNo1991} When performing computations, we must
first encode information onto physical systems and evolve those systems to carry
out the necessary computations. If we could identify a physical system that is
inherently resistant to noise and encode information onto it, we would be able
to construct a naturally fault-tolerant quantum computer. Furthremore, current
knowledge suggests that all physical interactions are local. Therefore, if we
could identify a system that is immune to local interactions, we could construct
a decoherence immune qubit.

Recall our definition of a topological invariant, as previously defined in
class. Both the Gauss-Bonnet theorem and the calculation of the Berry phase
require an integral over the entire system. Local information alone is
insufficient to determine the topological invariants of a system. Similarly, if
one has the ability to alter a system locally, they cannot alter the topological
invariant number of that system.

Using this as a guide, we turn our attention to topological systems. In the
limit of large system size, a topologically non-trivial phase can emerge from a
gapped quantum many-body system. One key feature of the non-trivial phase is
that it possesses topologically degenerate ground states that are not degenerate
in the topologically trivial phase. Therefore, the degeneracy of the ground
states is a direct consequence of the topology of the system's phase. Applying
the logic from the previous paragraph, we expect these ground states to be
protected from local noise, meaning decoherence should not affect them.

In conclusion, we can encode information onto these ground states and expect
this encoding to be immune to decoherence. This insight offers a promising
avenue for constructing a fault-tolerant quantum computer by leveraging the
properties of topological systems.

\section{Majorana Zero Mode: Physical Realization}
\label{sec:org89146da}
As an illustration, Alexi Kitaev proposed a one-dimensional chain of fermions in
which the spin is disregarded \cite{kitaevUnpairedMajoranaFermions2001}. The
Hamiltonian of the chain is given by \(H = \sum_{n=1}^{N} [- \mu
(a^{\dagger}_{n}a_{n}- \frac{1}{2}) - w (a^{\dagger}_{n}a_{n+1} +
a^{\dagger}_{n+1}a_{n}) + \Delta a_{n}a_{n+1} +
\Delta^{*}a^{\dagger}_{n+1}a^{\dagger}_{n})]\), where Kitaev discovered that
when \(|\Delta| = w > 0\), the system is in a topologically non-trivial
phase \cite{aliceaNewDirectionsPursuit2012}. The degenerate ground state can be
obtained through the following transformation, where \(\theta\) is the
superconducting phase and \(\gamma\) is the Majorana operator: \(a_{n} =
\frac{1}{2}(e^{-i \theta /2}\gamma_{2n} + e^{i\theta/2}\gamma_{2n-1})\) and
\(a^{\dagger}_{n} = \frac{1}{2}(e^{i\theta/2}\gamma_{2n} -
e^{-i\theta/2}\gamma_{2n-1})\). This transformation leads to the Hamiltonian \(H
= - i w \sum_{n=1}^{N-1}\gamma_{2n}\gamma_{2n+1}\), which can be easily
diagonalized to obtain its eigenstates by introducing new fermionic operators
\(\tilde{a}_{n} = (\gamma_{2n}-i\gamma_{2n+1})/2\). It can be observed that \(H
= 2w \sum_{n=1}^{N-1}(\tilde{a}^{\dagger}_{n}\tilde{a}_{n}
-1/2)\) \cite{huangIntroductionMajoranaZero2021}. As \(\gamma_1\) and \(\gamma_{2N}\)
do not exist in the Hamiltonian, they commute and have zero energy. Excitation
modes of these creation operators correspond to Majorana modes with zero energy,
hence the name ``Majorana zero mode''. The two Majorana modes can be combined into
one fermionic mode by \(\tilde{a}_0 = (\gamma_1 + i \gamma_{2N}) /2\). The
\(\ket{0}\) and \(\ket{1}\) states with respect to such fermionic excitation have
degenerate energy but are topologically protected. Therefore, they can be used
to encode decoherence-free qubits.

\section{Bibliography}
\label{sec:org456ef45}
\subsection{References}
\label{sec:orgfa8b0e2}
\bibliographystyle{apalike}
\bibliography{finalp}
\end{document}