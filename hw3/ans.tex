% Created 2023-02-26 Sun 15:45
% Intended LaTeX compiler: pdflatex
\documentclass[11pt]{article}
\usepackage[utf8]{inputenc}
\usepackage[T1]{fontenc}
\usepackage{graphicx}
\usepackage{longtable}
\usepackage{wrapfig}
\usepackage{rotating}
\usepackage[normalem]{ulem}
\usepackage{amsmath}
\usepackage{amssymb}
\usepackage{capt-of}
\usepackage{hyperref}
\usepackage{amsthm}
\author{Yusheng Zhao}
\date{\today}
\title{Homework 3 Answer}
\hypersetup{
 pdfauthor={Yusheng Zhao},
 pdftitle={Homework 3 Answer},
 pdfkeywords={},
 pdfsubject={},
 pdfcreator={Emacs 28.2 (Org mode 9.6)}, 
 pdflang={English}}
\begin{document}

\maketitle


\section{Problem 1}
\label{sec:org602696d}
Recall, Born-von Karman boundary condition, \(\psi(\vec{r}) =
\psi(\vec{r}+n_{i}\vec{a}_{i})\). For momentum eigenstates, we need \(e^{i
\vec{k}_{i}\cdot \delta \vec{r}_{i}} = 1\). Where we define \(\delta \vec{r}_{i}
= n_{i} \vec{a}_{i}\) and \(n_{i} \in [1,N_{i}]\).

To satisfy the condition, we need \((\vec{k}_{i} \cdot n_{i} \vec{a}_{i} ) =
2\pi * m, \forall i \in \{x,y,z\}\), where \(\vec{k}_{i} = h_{i} \vec{b}_{i}\). And
\(\vec{b}_{i}\), \(\vec{a}_{i}\) are basis vector in the momentum and real space
respectively and \(m\) is an integer.

Within the first Brillouin Zone, \(m=1\). We then have, \(\vec{k}_{i} \cdot
\vec{a}_{i} = 2\pi /n_{i}\). Recall \(\vec{k}_{i}\) and \(\vec{a}_{i}\) are
parallel. We see \(||\vec{k}_{i}|| = 2\pi /(||\vec{a}_{i}||n_{i})\) where \(n_{i}
\in [1,N_{i}]\). We see now, \(\vec{k}_{i}\) could take \(N_{i}\) different values.

The arguement above should be repeated in all three dimensions, in result
\(\sum_{i} \vec{k}_{i}\) could take \(\prod_{i}N_{i}\) different results which is
also the number of lattice points in real space.

\section{Problem 2}
\label{sec:orga628091}
Using \(f(\vec{r}) = f(\vec{r} + \vec{R})\), let's sum up and take average
\(f(\vec{r}+\vec{R})\) with \(\vec{R}\) range from all lattice points in our
crystal. It should be equal to \(f(\vec{r})\).



\begin{align*}
f(\vec{r}) & = \frac{1}{N_{crystal}} \sum_{R\in\{\vec{R}\}} f(\vec{r}+\vec{R}) \\
           & = \frac{1}{N_{crystal}} \sum_{R\in\{\vec{R}\}} V \int \frac{d\vec{k}}{(2\pi)^{3}} f_{\vec{k}} e^{i \vec{k}\cdot (\vec{r}+\vec{R})}\\
           & = \frac{1}{N_{crystal}}  V \int \frac{d\vec{k}}{(2\pi)^{3}} f_{\vec{k}}  e^{i \vec{k} \cdot \vec{r}} \sum_{R\in\{\vec{R}\}} e^{ i (\vec{k} \cdot \vec{R})}\\
           & = \frac{1}{N_{crystal}}  V \int \frac{d\vec{k}}{(2\pi)^{3}} f_{\vec{k}}  e^{i \vec{k}\cdot \vec{r}} \sum_{\vec{G}} \frac{(2\pi)^{3}}{v} \delta (\vec{k}-\vec{G})\\
           & = \frac{1}{N_{crystal}}  V/v \sum_{\vec{G}} f_{\vec{G}} e^{i\vec{G}\cdot \vec{r}}\\
           & = \sum_{\vec{G}} f_{\vec{G}} e^{i \vec{G} \cdot \vec{r}} \qed
\end{align*}

where we have used in the last line, \(V = v * N_{crystal}\). The total volume of
the crystal is the number of unit cells multiplied by the unit cell volume.
\end{document}